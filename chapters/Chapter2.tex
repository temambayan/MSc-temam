%\setcounter{chapter}{1}
\chapter{ Emission mechanisms and observational properties of gamma-ray bursts}
\label{chap:2}
\section{Introduction}
This chapter  comprises eight sections that  logically  related  to  explain  gamma -ray physics. Overall ,  the  phenomenology  of gamma - ray  bursts : emission mechanisms of gamma-ray bursts,  observational properties of  gamma-ray bursts (prompt and afterglow phases ), interpretation  of gamma-ray afterglow  especially x-ray afterglow , decaying of flux with time are  discussed  subsequently. Finally flux (F) and light curve intensity (f) are  defined  and their  equations  are  derived briefly.    
\section{GRBs production mechanisms}
GRB emission mechanisms:- are the theories or models that visualize or  explain  how the  energy from GRBs  progenitor ( sources ) is turned to radiation. In the early 1990’s   more than 100  potential  models  were  developed to describe the phenomenon of GRBs. However, more  constraining  observations  over  the  years have resulted in the  development of a standard  model called  " Fireball model " to describe  well about the  emission  mechanisms of GRBs and their afterglow , properties of GRBs progenitors. It was a neat theoretical standard model that has been revised that   attempt to explain  the mysterious events of GRBs for a longer time.\citep{13}
\subsection{ Basic fireball model }
Gamma-Ray Bursts (GRBs)  are  the  most  energetic  events  in  the  universe. During GRBs, the  most  powerful  energy  equal  to over  9000 supernovae  can impressively ejected.  These   energy  levels   are  so   extreme   that  they  cannot  be  created  by  thermal  processes. So, what  causes  these  high  energy  levels? The Fireball Model  is   one  of  the   few   models   that  has   been  put f orth  to  explain why  GRBs  tend   to   have  such   high  energy   levels. It  also  attempts  to explain  the  time  scales  that  govern  the   phenomenology  of  prompt and afterglow   phases.\citep{13} \citep{14}\\\\ 
The   variability  of  GRBs   light  curves   directly  related  to  the  high  energy  levels ,  as  the   variability   indicates  that  it  occurs  over  a very small    area  with  the   emission    of  GRB   energy  being  in  the order of $10^{52}$ ergs , coming  from   a very  small   volume  of   space   with  highly concentration   of  radiation   energy,  and   then   theorized  that   a Lorentz factor of $ \Gamma $ $ \sim $100  be  associated   with   the  GRB.In  short, the fireball  model   must  be  able  to  encompass   all  of  these  variables  in order  to apply to   all  GRBs   (and  thus  be   a plausible  model  to  study  GRB  physics.\citep{14}\\\\
The  name of the  fireball model  suggests the mechanism to which a GRB occurs in a fireball  of  ultra-relativistic  energy  consisting  of  optically  thin  material with  very  few  baryons. In  essence, during  the   GRB  event, the inner engine remains  undetectable  due  to  the  optical  thickness  and  the lack of a thermal profile due  to  the  compactness  problem .The  highly relativistic expansion of fireball  overcome  the  compactness  problem  and  can cause  the internal shocks that  produce  the  detectable  GRB, while  the  external  shocks  form  the gradual afterglow \citep{15}.\\\\
 The  first  relativistic  fireball  model  was  proposed  by (Paczynske 1986,Goodman 1086).They  had  shown  that  the  sudden  release   of a large  quantity  of gamma ray  photons  into  a compact  region   can  lead  to  an  opaque  photon–lepton “fireball” through  the  production  of  electron–positron pairs  $ e^{-} e^{+} $. The  most  fundamental  property  the  fireball  can  be  characterized  by its initial  energy $ E_{o}$. In  the  fireball  there  are $ M_{o} $  baryons  (electrons  have  negligble  mass)  with $ M_{o} \ll\frac{E_{o}}{c^{2}}$ , and its mean energy per baryon, $\eta = \frac{E_{o}}{M_{o}c^{2}}$ \citep{14, 15}.\\\\
Fireball  model  predicted that  when the expanding plasma becomes optically thin ,  and hence the  emitted  radiation  escapes   within  created  burst . This  mechanism would  generate  a quasi-thermal  spectrum  rather  than  the observed power-law spectra, thus  indicating  the  difficulty inherent to explaining the duration of the GRB  having  such  a small  timescale  (just a few seconds). Moreover, the fireball baryonic  load  is  another  model  which  converts  all its energy into kinetic energy  rather  than  into  luminosity to  produce  a quasi-thermal  spectrum. This model,  however,  does not explain the efficient emissions  of radiation. In particular,  the  origin  of  the  emission  associated  with  the  two  phases is produced  by  two different  mechanisms:  a matter-dominated, and  a radiation-dominated. The  assumption  that  the  fireball is  matter-dominated  is  widely  used, and  which  consists  of  baryons ,  electrons  and  positrons , and  photons  resulting  from  the merger  of  binary  neutron  stars  or  a collapse  of  massive stars .\citep{10} \citep{15}\citep{16}\\\\ 
\begin{figure}[h]
\begin{center}
\includegraphics[scale=0.5]{Figures/fig5.png}
\caption{Visualization  of  the  fireball  model that illustrates  the various steps of  basic  standard  model  with  the internal  and  external  shocks and  radiations. At  the left  the  two main  scenarios ( collapser and  merger ) indicated  that they  lead  to  a central  black  hole  surrounded by a disk. (from Gehrels et al. (2002), credit Juan Velasco)\citep{13}}
\end{center}
\end{figure} 
The  emitted  energy  is  higher than  the  mass of  the baryon in the rest frame by a factor of  $\sim $ 100, with  the  baryon  accelerated  with in an expanding fireball to a higher Lorentz factor, $\Gamma $. During this process, two major out comes can be seen: at the photosphere, a fraction of the thermal energy is radiated away  and  the  accelerated  electrons  produce  a non-thermal  gamma ray spectrum by a synchrotron  or an IC  processes in the  internal shock at large jet radius. Rather, the outflows that formed  from the central engine are believed to be dominated by Poynting flux.\citep{ 15}\citep{16}\\\\
The shocks in the fireball model are collisionless, whereby the particles involved are accelerated and scattered within the Fermi process when crossing the shock interaction. This can result in the type of energy distribution that can be described by a power law ( $ \alpha $  $ \sim $ 2 - 3 ). In such a situation, the electrons emit a non-thermal radiation of photons via two different mechanisms, synchrotron and Inverse compton scattering, that extend to very high energy (GeV bands).\citep{16} \citep {17}. 
\subsubsection{Disspative process}
Disspative process :- is the  of outflows or shock waves from central engine  interact with interstellar medium ( ISM ) to produce both GRBs and its afterglow - the external and internal shock models, that  successfully interpret the prompt and afterglow emissions respectively.In particular, the origin of the emissions associated with the two phases is produced by two different processes. \citep{8} \citep{17}.\\\\
\textbf{Internal shock model}\\
 Internal shock  model :  is  the  model  that    highly   obseved   energetic  gamma -rays  flash  produced  and   explained  by fireball model.  Immediately  after the initial GRB event, shock waves emanate from the inner engine  at  relativistic  speed   (   99.995\%   of  the speed of light ) and  a Lorentz factor of $ \sim 100 $. The fireball is  dynamic  object  that contains  a  mixture  of charged  particles , photons , and  magnetic  fields  known   as  plasma , in which the particles move  fast with  random   motions . In the context of early  evolution  of a GRB , the fire ball  carries  the  energy  deposited  near  the  central engine  and  expands outwards  radially. During  expansion  several  shock  waves  are  emanating from the compact source , and   are  traveling  at  different  relativistic  speeds . Thus , the  interaction  between  different  shock  wave  fronts   cause   energetic gamma-ray emissions. \citep{17}\citep{18}\\\\
The internal shocks  traveling  at relativistic  speeds  convert  kinetic energy into gamma-ray  photons, this is  the only  way to  get high  energy gamma-rays that are observed (as previously mentioned, they cannot be emitted through a thermal process). When  internal shocks interact with  each  other as they  are  moving  at  relativistic  speeds, the  interactions  produce  Inverse  Compton and Synchrotron emissions.fig 2.2 \citep{18} 
\begin{figure}[h]
\begin{center}
\includegraphics[scale=0.3]{Figures/fig6.png}
\caption{Standard fireball model.}
\end{center}
\end{figure}
Initially, the fireball is optically thick but as it expands and cools down and  becomes optically thin, allowing the gamma-ray photons to escape. Early models had the fireball and the internal shock waves as being purely radiative, but this didn't follow what was being observed (it would have made a profile too smooth). To solve this problem, some baryonic mass was added. This allowed for the internal shocks to become effectively contaminated. The added baryonic mass also aids in the conversion of some radiation energy into kinetic energy, which helps with an added kick to the relativistic kinetic energy of the shock waves, this in turn increases the gamma-ray energy more,Even if all of the shock waves emanate from the core at the same speed they will eventually cross over multiple times. As the shells are emitting through inverse Compton, it is slowing the shock front, thus increasing the times that many shock waves interact with one another. The earlier shock waves are likely to be emitted slower than the later  shock waves, this would also increase the amount of interactivity between the different shock waves.\citep{16}\citep{18}.\\\\
\textbf{External shock model}\\
The external shock waves are used to explain the afterglow that was first detected by BeppoSAX in 1997, as the internal shock waves are not able to explain the duration of the afterglow nor the wavelengths that are detected (which range from soft x-ray through to radio). The name can be a little misleading at first; the external waves actually refer to the internal waves at a later stage --once they've cooled down and continue emanating from the source. As the shock waves continue out they will eventually interact with the Interstellar Medium [ISM] (such as a molecular cloud or some other impedance), and it is the shock waves' interaction with the dust/gas that cause the afterglow . Unlike the internal shocks, the external shocks are primarily a thermal emission(see fig 2.3). The energy transferred from the shock waves is deposited into the ISM; this material can then be caught up in the shock front and emit radiation. As the shock waves began with a lot of energy, there is a lot that can be deposited into the ISM, this is what can cause such long afterglow and why it covers all parts of the energy spectrum .\citep{19}\citep{20}.
\begin{figure}[h]
\begin{center}
\includegraphics[scale=0.3]{Figures/fig7.png}
\caption {schematic view of the structure of the relativistic jet produced by the gamma-ray burst. The external shock arises as a result of the impact of the jet on the stellar wind of the progenitor. This is where the final goodbye of the SOS
similar emission from the collapsing star forms, which is characterized by a smooth (but non-monotonic) light variation. The internal shock persists as long as the central engine continues operating this is where rapidly varying gamma-ray , x-ray  and optical radiation formed.\citep{21}}
\end{center}
\end{figure}
A relativistic materials/jets are running into  external ambient medium or stellar wind. In each time, the ejecta run a high density environment in which they produced a peak in the mission called external shock . In the external shocks, the jets may be forward shocked or reverse shocked.As the material in the jet expands, accelerates and compresses interstellar medium,It creates a forward shocks. The deceleration of forward shocks is occurred when the rest mass energy of the swept up particles equal to the ejected energy. This sets a deceleration length scale at ($\sim  10^{16} $ cm) . The reverse shock is formed by the deceleration of the jet material and propagates back into the relativistic flow. This happens when the rest mass energy of the swept up particles is greater than the ejected energy .\citep{20}\\\\
Although it would be correct to assume that all GRBs have an external shock, about half of detected GRBs don't have a detectable afterglow. The reason that no afterglow  being detected is not thought to be the exposures aren't long enough, or because we're observing too early or too late. Rather, GRBs occur in high mass systems, whether it be through a supernova or NS-NS and NS-BH merges, this means that they had very short stellar lives and may still be inside of a molecular cloud. Molecular clouds are very optically thick environments so the reason we're not able to detect the afterglow in about 50\%  of the time could just be due to reddening, absorption, or scattering.\citep{20, 21}
\subsubsection{Radiative process}
\textbf{ Synchrotron Radiation}\\
Synchrotron emission is the non-thermal radiation produced when a relativistic
electron gyrates in a uniform magnetic field. Synchrotron radiation can explain
the GRB prompt emissions, and is considered to be one of the more important
mechanisms in various astrophysical phenomena.  synchrotron shock mechanism,
which is produced by the optically thin plasma in a weak magnetic field, can be
used to predict the form of the observed spectra \citep{1}\citep{15} \citep{18}\\\\
Synchrotron emission can be classified as having two regimes: the "fast-cooling"
phase,which describes when the timescale for the cooling of the electrons is shorter
than the dynamical lifetime of the source, resulting in an electron that cools quickly compared to the low-level injection of energy; conversely, "slow-cooling" occurs when the timescale for the cooling of the electrons is longer than the dynamical lifetime of the source . The differences between these two regimes are associated with the emission’s radiative timescale \citep{9} \citep{10}.\\\\
The peak frequency, the cooling frequency, and the self-absorption frequencies set
the characteristic break frequencies in the synchrotron spectra. These frequencies
evolve with time;indeed, their evolution is reflected in the complexities observed in
the shapes of the light curves at certain band energies . This model can successfully
describe the afterglow. Thus, the optically thin synchrotron spectrum is currently
considered the best spectral fitting model for most GRBs. The first synchrotron  model was applied to the spectral fitting of GRBs by Tavani (1996), and subsequently  by Baring and Braby (2004)\citep{15}\citep{22}\\\\
\textbf{Synchrotron Self-Compton}\\
Inelastic collisions between low-energy photons and ultra-relativistic electrons are
known as the IC processes. Each astrophysical source has an Synchrotron Self-
Compton (SSC) scattering component when synchrotron radiation that energizes
it provides the means to scatter its seed photons to high energies and across a large
frequency range. Thus, the phenomenon responsible for creating high-energy emissions from GRBs and other astrophysical sources is accepted to be the SSC mechanism. The SSC mechanism, while complex, uses a simple power-law function to explain the injected electron spectrum. The SSC spectrum can be described precisely as carrying out a complicated seed photon spectrum convolution and electron energy distribution. In certain circumstances, the GRB spectrum can be modelled as an SSC component at very high energy $ \sim $10 MeV.\citep{11}\citep{16} 
\subsection{ GRBs progenitors models }
There are three  models  to produce lots of energy in nature: nuclear, gravitational, and rotational. The nuclear energy does not have enough efficiency to power a GRB. For  example, the proton-proton  chain  that is responsible for energy production in stars  has  an  efficiency  of  about  0.0067 which was not enough to power a GRB. However , the last two mechanisms ( gravitational and rotational ) are playing   greate roles powering  GRB energy release.
\begin{figure}[h]
\begin{center}
\includegraphics[scale=0.6]{Figures/prog of GRBs.png}
\caption{Schematic scenarios for plausible progenitors of long and short GRB.  Long GRBs come from the collapse of massive , while  rapidly rotating stars and short GRBs result from the merger of compact objects.\citep{23}}
\end{center}
\end{figure}
The central engine that produces the initial energy, $ E_{o} $ , is hidden from direct observation. However, the observed temporal structure is thought to reflect the  activity  of  the central engine.\citep{23}\\The central engine must satisfy the following general features:\\
• Capable of producing an extremely relativistic energy flow containing $\approx$ $ 10^{51} $ - $ 10^{55} $ erg.\\
• Highly variable flow resulting in highly variable light curves.\\
• Its  activity can last from a fraction of a second to a few hours.\\
• Possibility of late time activity that may cause X-ray flares.\\
• Relatively rare event as suggested by observed GRB rates.\\
\subsection{Working mechanisms of central engine}
The central  engine is of great importance, as it needs to be able to push material out very near the speed of light. The inner engine of a GRB is a highly compact source, and it is the highly compact nature of this object that leads to the idea that the core of the inner engine of a GRB is either a neutron star or a black hole (as they're the two most compact sources that we're currently aware of )
The workings of this inner engine will alter depending on whether it is a long or short GRB being observed. A short GRB has been thought as occurring during a neutron star binary collision [NS-NS] or a neutron star-black hole collision [NS-BH]. It has been suggested that a long GRB could be associated with a hypernova when  a Wolf-Rayet type star undergoing a core collapse supernova.\citep{21}\citep{23}
\subsection{Supernova}
Supernovae (SNe) are  highly  powerful   explosions  that  terminate  the  life of some  stars. The  study of SNe  was  initiated  in  early  1930s  by W. Baade and F. Zwicky , and  the  first  bright  supernova (SN 1987A ) was  reported  in  1987. It was  suggested  that , the  source  of  enormous   quantity  of  energy  released in  SNe  is  the   gravitational  collapse  of  a star  to   neutron  star. During explosion , some  solar  masses  are   ejected  in  the  interstellar  space  with  a kinetic  energy  of  the  order  $ 10^{51} $ erg.  The  ejecta  contains  heavy  elements  that  are  vital  for  the  chemical  evolution  of  galaxies, stars, and planets. SNe  may  be   sources  of  cosmic  rays , and   Some  compact  remnants such  as   a neutron  star  or  a black  hole.\citep{24} \\\\
According to observational  properties , supernovae   explosions  are  classified  as  Type I and  Type II.  where the former and the latter characterized by absence  and presence  of hydrogen lines in  the  spectrum  respectively. Each type has its own  characteristic  light  curve, although  a wide  variety  of  deviations from the general shape is detected, resulting from individual properties 
\begin{figure}[h]
\begin{center}
\includegraphics[scale=0.6]{Figures/SN.png}
\caption{Supernova light curves}
\end{center}
\end{figure} 
Type I  supernova  arises  When white dwarfs are  collapsed  and  reaching   the Chandrasekhar  limiting  mass by accertion where as  Type II  supernovae  that  are  associated  with  the  collapse  of iron cores  of  massive  stars. These  stars  have  large  hydrogen  rich  envelopes; hence  the  evidence  of  hydrogen  in  the  spectrum.  If  a supernova’s  spectrum contains  lines  of  hydrogen  it  is  classified  Type  II; otherwise  it is Type I. As  massive  stars  evolve  much  more  rapidly than low mass  stars, old  stellar populations, where no star formation occurs, have  outgrown  the  Type II supernova stage  ( see fig 2.5 above ). \citep{24}\citep{25}\\\\
According to  the observed  features  at  early spectral explosion , Type I SNe further  classified  in  Type Ia,  Ib  and  Ic.  Type Ia  supernovae are characterized  by  the  presence  of  a clear ( Si II  )  absorption  line  around 6150 Å  and   their  late   spectra   show   many  lines  associated  with  Fe  emission , while Type Ib and Ic  supernovae  do  not show  this  ionized  silicon (Si II)  absorption  line and  are  distinguished  according  to  the  presence  or not, respectively, of moderately  strong  He I  lines  around 5876 Å.\citep{26}\\\\
Type II SNe  can  also  be  sub-divided  based on their spectra. and  they   show  very  broad  emission  lines  which  indicate  expansion velocities  of  many  thousands  of  kilo-meters  per  second  have   relatively narrow  features  in  their  spectra.  These  are  called  Type  IIn,  where  the ’n’ stands  for  ’narrow’.  They  can  potentially  be  produced  by various  types  of  core  collapse  in  different  progenitor  stars,  possibly  even by  type  Ia  white  dwarf  ignitions ,  although  it  seems  that  most  will  be from  iron  core  collapse  in  luminous  super giants  or  hyper giants. The  narrow spectral  lines  for  which  they  are  named  occur  because  the  supernova  is expanding  into  a small  dense  cloud  of  circumstellar  material.\citep{27}\\\\
The  other  sub  division supernova  is  ”Type IIb”  that   used  to  describe  the  combination  of features  normally  associated  with  Types II  and  Ib. These  supernovae , like those  of  Type II , are  massive  stars  that  undergo  core  collapse. However the stars  which  become  Types  Ib  and  Ic  supernovae  have  lost  most  of  their outer ( hydrogen )  envelopes  due  to  strong  stellar  winds  or  else  from interaction  with  a companion .\citep{26}\citep{27}
\subsection{GRB-SN assocition}
The early 1980s flashes or bursts of $ \gamma $ - ray light were detected with space probes from diverse directions in the sky. It then also became known that the earlier US military satellites Vela (1967- 1984) , designed to detect the $ \gamma $ - ray flash from atomic bomb explosions , had seen such bursts. With the Burst and Transient Source Experiment (BATSE), on board of the Compton Gamma Ray Observatory (CGRO), launched in 1991, many more such flashes were detected.\citep{27}\\\\
 In the late 90s it was realized that, in addition to the  normal core collapse and thermonuclear explosions, there are more energetic supernovae  with an energy output and $ 10^{45} $  J, i.e., they are at least 10 times as energetic as a normal  supernova that is 5 - 50×$ 10^{44} $ J. These are now often referred to as hypernovae (HNe) or alternatively as broad lined supernovae, since they have very broad lines. SN 1998bw  was also observed as an LGRB (long gamma ray burst), establishing the first connection between a GRB and a supernova, the death of a massive star.         \citep{25}\citep{27}\\\\
In addition, many GRB afterglows show bumps in the light curve that are consistent with  an underlying hypernova like event. Interestingly, at present all GRB supernovae are classified as SNe Ic, i.e., supernovae that have lost both their hydrogen and their helium envelopes . Hypernovae are produced by more than one type of event: relativistic jets during formation  of a black hole from fallback of material onto the neutron star core, the collapsar model,or during the last phases of the coalescence of neutron star binaries. These catastrophic events are believed to exist at the central engine of highly energetic gamma-ray bursts \citep{28}\\\\
 Stars with initial masses between about 25 and 90 times the solar mass of the sun develop  cores large enough  to create a black hole  after a neutron  star formed by  supernova explosion  and some material will fall back onto it. In many cases this reduces the luminosity of  supernova, and above 90 $ M_{\odot} $ the star collapses directly into a black hole without a supernova explosion. However, if the progenitor is spinning quickly enough the in falling material generates relativistic jets that emit more energy than the original explosion.\\\\
 In some cases these can produce gamma ray bursts, although not all gamma ray bursts are from supernovae. Some black hole binaries appear to be linked to the hypernovae believed to power gamma - ray bursts .  In our GRB the materials are fallback but not a collapsar because the amount of falling mass is small and it don’t have to be collapse in to Black Hole.\citep{29}
\section{Observational properties of gamma-ray bursts }
GRBs composed of two main radiative phases: the prompt and afterglow phases.The former typically observed in soft gamma-ray (10 keV to 10 MeV), and generally  lasts between $ \sim $ 100ms and $ \sim $ 1000 s, although there is a wide variety of different temporal behaviors observed,from single pulses to complex temporal evolution. The spectrum of this prompt emission is non -thermal, often described by a Band function with a typical peak around $ \sim $ 200 keV.
The afterglow emission is most often detected from X-rays to radio waves and fades  away  with time. In the optical, the temporal fading goes typically as $ t^{-1} $ , however the slope of this fading depends on the wavelength and on the burst. This means an afterglow observed in the optical will frequently fade beyond  the  reach  of  most  ground - based  telescopes  within a week. In the  radio however, there is evidence of  emission from the afterglow up to a few months or even years after the burst. Sudden flash of gamma rays emitted in the creation of a GRB was the route by which GRBs were detected and the property for which they are named. The  prompt emission is readily detected, even by rudimentary space-based  gamma - ray detectors, due to the extreme high-energy photon budget that GRBs exhibit.Indeed, at peak, GRBs outshine all other sources within gamma-ray sky, including the Sun.\citep{18}\\
\subsection{ Prompt emission of  gamma - ray  bursts}
 prompt emission of GRBs is defined as the emission observed during  gamma-/hard X-rays phase, whose photons are triggering the space instrumentation leading to multi-wavelength follow-up observations.	It is believed to be the direct outflow ejected from the central engine; as per the "fireball" model , deposits its gravitational energy into a thermal explosion. In other words, the prompt emission occurs when the kinetic energy from a catastrophic explosion event, such as massive star core collapse or the merger of two compact stars, is converted into electromagnetic radiation due to the internal shocks that result from collisions between shells of ejecta.\citep{23}\\\\
prompt emission generated due to internal shocks magnetic dissipation within the  fireball take place effectively above the pair production (  photosphere at $ 10^{12} $ to  $ 10^{14} $ cm ).These  shocks splits from mini-shells within a jet produced by unsteady accretion of  materials  onto black hole or by the merger of binary neutron stars (BNS). The shells have a distribution in lorentz factor $ \gamma $ $ \propto $ $ \Gamma $ where $ \Gamma $ is bulk lorentz factor.\citep {22}\citep{23}\\\\
\begin{figure}[hpbt]
\subfloat[]{\includegraphics[scale=0.4]{Figures/collapser.png}}
\subfloat[]{\includegraphics[scale=0.4]{Figures/fireball.png}}
\caption{(a) Collapser  model  shows internal and external shocks producing prompt and afterglow emissions respectively. \\ (b) Schematic evolution of the jet Lorentz factor and examples of symbolic locations of radii : the saturation radius $r_{s} $ , photospheric radius $r_{ph} $ , internal shock radius $ r_{is} $ and external shock  radius  $ r_{_{es}} $}
\label{GRB prompt emission}
\end{figure}
Ground-based facilities have detected radiation up to a trillion times the energy of visible light from a cosmic explosion called a gamma-ray burst (GRB). This illustration shows the set-up for the most common type. The core of a massive star (left) has collapsed and formed a black hole. This “engine” drives a jet of particles that moves through the collapsing star and out into space at nearly the speed of light. The prompt emission, which typically lasts a minute or less, may arise from the jet’s interaction with gas near the newborn black hole and from collisions between shells of fast-moving gas within the jet (internal shock waves). The afterglow emission occurs as the leading edge of jet sweeps up its surroundings (creating an external shock wave) and emits radiation across the spectrum for some time — months to years, in the case of radio and visible light, and many hours at the highest gamma-ray energies yet observed. These far exceed 100 billion electron volts (GeV) for two recent GRBs. Credit: NASA’s Goddard Space Flight Center\\\\
In the region around $ \sim $ $ 10^{12} $ cm to  $ 10^{14} $ cm , the collisions between different parts of the flow is produced in different shells (see fig 2.6 (a) and(b)). As a fast shell catch up with a slower ones, they form strong internal shocks that propagates in both shells with out deceleration. Once shell became above the photo sphere,the heated and accelerated electrons cool by synchrotron emission then radiation is observed in $ \gamma $-ray band. Each collision that occurs above pair photo sphere produces a pulse in the GRB’s light curves \citep {22}\\
Thus, GRB light curves represent the count rates/photons recorded by the high energy detectors as a function of time. Each of the recorded events shows different variability patterns, meaning that each light curve is different from the rest. As  shown in Figure 2.7 , the light curves can be classified into four  different categories Pe’er, 2015:\\
• Single-peak events ( e.g. GRB 910711 ) \\
• A smoothed light curve composed with several peaks ( e.g. GRB 920221 )\\
• Separated multi-collisions (  e.g. GRB 930131A  ) and \\
• Irregular peaks (  e.g. GRB 991216  ) \\
\begin{figure}[h]
\begin{center}
\includegraphics[scale=0.4]{Figures/prompt.png}
\caption{Diverse light curves of the GRBs prompt emission detected by BATSE instrument.This sample includes short and long events.}[Firaol Fana]
\end{center}
\end{figure}
The result is that the fireball expands due to the effects of thermal pressure and is then accelerated to relativistic speeds, where the thermal energy is ultimately  released in the form of photons at the photo sphere. In the internal shock case,
the dissipation happens inside the ejecta, where the ejecta is decelerated by the  surrounding medium and this deceleration happens after the internal shock phase ceased. \citep{6 }\citep{23}
\subsection{Afterglow  emission  of gamma - ray  bursts}
Afterglow gamma - ray burst studied  extensively since the launch of swift.Afterglow  emission is created by the collision between ejected bursts and the surrounding medium or interstellar gas or dust , and fading slowly at longer wavelength. The GRB itself is rapid, lasting from less than a second up to a few minute at most. Once it disappears, it leaves behind a counterpart at a longer wavelengths from X-ray to radio bands . Then, they are remain detectable for day or longer. As we have mentioned above, afterglow emissions are dominated by external shocks. Due to luck of advanced instrument, early searches were unsuccessful largely to observe the bursts’ position at a longer wavelength immediately after the initial burst,Once the GRB faded deep imaging was able to identify a faint, distance host of galaxy at a location of GRB as pinpointed by the optical afterglow.\citep{15}\citep{22}\citep{23} 
\section{Interpretations of afterglow gamma- ray bursts }
Before launch of the Swift satellite, broad-band, late time (t > $ \sim $ 10 hours) afterglow data had been collected for a moderate sample of GRBs. These observations were generally consistent with predictions of the external forward shock,synchrotron emission, model. The main observational properties of late time afterglow radiations are:\\
• In general the optical afterglow displays a power law decay behavior $ F_{\nu} $ $ \propto $ $ t^{-\alpha} $   , with a decay index $ \alpha \sim $ 1. This is consistent with the prediction of the standard external shock afterglow model.\\
• A temporal break in the optical afterglow light curve is usually detected for bright GRBs. The break time is typically around a day or so, which is followed by a steeper decay with slope $ \alpha  \sim $ 2. This is consistent with the theoretical prediction of a “jet break”.\\
• The radio afterglow light curve initially rises and reaches a peak around
10 days, after which it starts to decline (e.g. Frail et al., 2000). The peak
usually corresponds to passage of the synchrotron injection frequency $ \nu_{m} $ ,
or the synchrotron self-absorption frequency $ \nu_{a} $ , through the radio band.\\
• The broad-band afterglow spectrum can be fit with a broken power law, at
a fixed observer time as one expects for the synchrotron afterglow model.\\
• For bursts with high-quality data richer features in the optical light curves have been discovered, which include bumps and wiggles that deviate from the simple afterglow model predictions. Smooth bumps in afterglow light curves with duration $ \delta  t _{obs}  \sim  t_{obs} $ may be interpreted as due to density bumps in the external medium  where as sharper features in light curves might be due to energy injection from the central engine  angular fluctuations in energy per unit solid angle \citep{18}.
\subsection{ Early time afterglow }
\subsection{ Late time afterglows }
Before swift mission, afterglow observations was started after several hours ( 10 hrs) after bursts trigger. The optical afterglow of late time displays a power law decay behavior $ F_{\nu}   \propto   t^{-\alpha} $  , with a decay index  $\alpha  \sim  1 $ . The temporal break in  optical afterglow light curve was detected for bright GRBs. The break time is typically around a day and followed by steeper decay with decay slope of $ \alpha  \sim 2 $.
The radio afterglow light curve initially rises and reaches a peak around 10 days
after  starts to decline . The peak usually corresponds to the passage of synchrotron injection frequency $ \nu_{m} $ or synchrotron self absorption frequency $ \nu_{a} $through the radio band. The broad band afterglow spectrum can be fit with a broken power law at a fixed observer time. \citep{24}\citep{25}
\section{ Theoritical  interpretation  of  X-ray  afterglow }
One of the breakthrough discoveries by Swift was the X-ray afterglow behavior in the first few hours after a GRB, which were missed by Beppo-SAX. Swift revealed several striking features of x-rays : (i) many early X-ray light curves show a canonical behavior with three distinct  power-law  segments (marked I - III in Fig 2.8 right), in some cases also a jet break at later times (IV); (ii) in about half of the GRBs, bright flares in the X-ray light curves are observed long after the end of the prompt
phase ($ 10^{2} $ s - $ 10^{4} $ s). In some extreme cases, these flares have integrated energy similar to, or exceeding, the initial burst of gamma rays, and severely challenge current theoretical models.\citep{26}\\\\
Afterglow gamma-ray bursts  observed at all wavelengths such as: X-ray , optical ,  IR, and radio . X-ray afterglow is the first and strongest, but shortest signal.In fact it seems to begin already while the GRB is going on. X-ray light curve observed several hours after the burst can usually be extrapolated to the late parts of the prompt emission. Due to  its low variability and observed time range (from minutes to weeks after the GRB event), a canonical X-ray light-curve for the afterglow was defined from the result of Swift /BAT-XRT instruments.\citep{27}\citep{28}\\\\
 The four  segments, with  their  corresponding  temporal  indices , are  associated two   by  two  and  identified  as  early   and late   afterglows . segments  I  and  II  are  the  steep  and s hallow  decay  respectively  , while  segments   III and  IV   respectively  a standard   afterglow  and   a jet  break . Part I  and III, marked  by   solid  lines, are most common   and  the  other two  components, marked by  dashed  lines , are  only  observed  in  a fraction  of  all bursts. Part I, thought  to  be  associated   with the  prompt   phase   when   the   central  engine  is  still  active; the rest of the  afterglows  are  due  to  the  dynamics  of  the  interaction  between  the jet and  the  surrounding   medium. \citep{30}\citep{31} \citep{32}\\\\
\begin{figure}[hpbt]
\subfloat[]{\includegraphics[scale=0.3]{Figures/X-ray Lc.png}}
\subfloat[]{\includegraphics[scale=0.3]{Figures/Lightcurve.png}}
\caption{(a)  Canonical GRB light curve , showing prompt phase  followed by afterglow phase.\\ (b) Canonical x-ray light curves with its  components: a steep decay phase (typical index of 3) which can then break to a shallower decline (shallow decay phase), a standard afterglow phase (pre-jet break phase), and possibly, a jet break and post-jet break phase. Sometimes an X-ray flare is seen.}
\label{Fg: Theoritical interpretation of X-ray afterglow}
\end{figure}
GRBs early afterglows detected within less than 100 seconds after trigger
in the swift mission. The canonical X-ray afterglow light curve generally includes
four phases such as early time steep decay phase, the shallow decay /plateau
phase, normal decay phase and late steep decay phase \citep{33}.\\\\
\subsection{Early  steep  decay afterglow x-ray light curves}.
This phase is the tail of prompt emission that governed by curvature
effect, for which emission from different viewing angles reaches the observer with
different delays due to the light propagation effects \citep{33}. The relationship between temporal and spectral slopes of higher latitude emission is $\alpha  = 2 + \beta $. It is independent of any of the environmental or other parameters such as peak frequency and cooling frequency that affects the closure relations for the external shocks. \citep{34}\\
Swift answer the debate of separation between prompt emission and late afterglow
regarding to internal and external origin of the prompt emission i.e internal
shocks are the origin of prompt emission.\citep{33} \citep{34}  As shown in (Fig2.8 above ), slope of early steep decay is around $3 < \alpha_{1} < 5$.
This phase may be simply the high latitude emission associated with the prompt
gamma-ray sources at $ R \gtrsim 10^{15} $ cm when the central engine turns off faster than the decline of the X-ray light curves . On the other hand, if the emission region is at much smaller radius than the rapidly declining X-ray light curve reflects the time dependence of central engine activity.\citep{35}\citep{36}\\
Detailed analysis of a sample of GRBs suggests that the high latitude ”curvature
effect” model can explain the early steep decay phase \citep{37}. As we have shown in
(Fig 2.8), the achromatic change of phases for sample GRBs indicates the light
curves transition. These GRBs followed the decay power law relation $ F_{\nu}\propto  t^{-\alpha_{1}} $ where, $ \alpha_{1}  = 2 + \beta $ for curvature effect model. Generally, this phase has already stayed between the time interval of $ 10^{-2}  - 10^{2} $ seconds and $ 10^{2} - 10^{3} $ seconds that  presented (in fig 2.8) at the right side.\\\ 
\subsection{Shallow /plateau decay X-ray light curves }
The shallow  phase  sometimes called plateau phase and characterized by very small decay with value of decay  $0.5 < \alpha_{2} < 1.0 $. It rises when the energy ejected to the decelerated external shock. When the energy is terminated, the decay of light curves become slow down and the transition to phase III (normal decay) is occurred . In this phase, the shape of light curves in the X-ray and optical bands should be similar where breaks occur at the same time in these bands.\citep{38}\citep{39}\\\\
 Behind  shallow / plateau  phase   there   are  two  acceptable  explanations:  
(1) There is  a smooth and gradual energy injection  arrives in the forward shock ,  due to the decrease of the Lorentz factor $ \Gamma $ at the end of prompt emission. The mass injected to the forward shock is the function of its lorentz factor and the energy injected. As a result $ \Gamma $ increases monotonically with radius  where as  the flux decays as a power law and depends on the mass and the energy injected .\citep{40} \citep{41}\\ 
(2) The central engine of the source stays active for hours after the burst and injects the smooth and continues energy at later times, several times after the burst. X-ray plateau resulted  from  the contribution of prompt X-ray emission scattered by dust in the host galaxy. \citep{41}.\\\citep{42}.\\
\subsection{normal decay of x-ray light curve}
The  normal  decay  is  the  III   phase  in  the  canonical  x-ray  afterglow. It   has a decay  slope  around $1.0 < \alpha_{3} < 1.5$ which was expected before swift and  consistent  with  standard fireball afterglow model in ISM [56]. Its  explanation  is  related to the end  of energy injection at the external  forward shocks. This situation happend  when :\\
(1) The  Lorentz factor is  falling  up to the point of minimal Lorentz factor that carries a significant initial energy in  external for ward shock.\\ 
(2) The  central engine needs to be  inactive. In general the normal decay is expected in the standard forward shock.\citep{43}
\subsection{Late steep decay following the plateau in X-ray light curves}
 The  IV phase  represented at the left side of fig 2.8  with  the decay slope  greater than 2. After the normal decay, X-ray emission is powered by a continues jet from a long lasting central engine. Then X-ray flux from the external shock is buried beneath this emission [58]. Indeed, the canonical X-ray light curve can be matched with the accretion history in the collapsar GRB model. This model assume that the X-ray luminosity is proportional to the accretion power of the central engine [59].
This late steep decay of swift, represents an achromatic steepening that happens
due to the jet breaks. When the Lorentz factor of the ejecta becomes larger than
$ \theta_{0}^{-1} $ compared to the jet opening angle $ \theta_{0} $ , the ejecta is collimated into a jet break. Finally, this phase is expected in the forward shock model as a jet break. Jet breaks are thought to happen due to the beaming of  emission from GRBs. This phase has pre-jet-break phase and post-jet-break phase, (see in Fig 2.8).\citep{44} \citep{45}
\subsection{Time breaks in swift X-ray afterglow}
As shown in ( Fig 2.8 ), there are three break points and the time at that points are called breaking time of afterglow light curves. These break times are the first break time, the second break time and the jet break time.\\
\subsection{The first break in the light curve ( $ t_{break,1} $ )}
It  is  the time at which  the  phase  of light curves  changes from phase I to phase II is took places. As it has been shown  in ( Fig2.8 ), the $ t_{break,1} $ is around  $t_{break,1} (10^{2} - 10^{3} ) $ s  < $ t_{1} $ <  $ t_{break,2}( 10^{3} - 10^{4})$ seconds).\\
The first break time is also the time when the slow decaying emission from the
forward shock become dominant over the rapidly decaying flux from the prompt
emission at a large angle. In sharply decaying flux, the prompt emission initially
dominates over the external shocks at t > $ t_{break,1} $\citep{46}.
\section{Flux decay with time of observed light curve }
The fluence (S) is the total radiant energy collected from the GRBs per unit area over the duration of the event (i.e., $ T_{90} $ ). It is computed by integrating its energy flux over time and the energy range of the detector (i.e., the total energy collected per unit time and per unit area). The fluence measured between energies $ E_{min} $ and $ E_{max} $ is given by[Firaol Fana]
\begin{equation}
S= T_{90}\int_{min}^{max}E\frac{dN}{dE}dE
\end{equation}
The energy flux of a burst is defined
\begin{equation}
F=\int_{Emin}^{Emax}E\frac{dN}{dE}dE
\end{equation}
When relativistic, conical and optically thin source moving with a Lorentz factor     $ \Gamma $ turns off abruptly, the flux declines rapidly with time [47]. In such type of source which is specified with spherical co-ordinate (r, $\theta $,  $ \varphi $ ), the source turned off at $ r =  R_{0} $. where r is the radius of the photo/jets, $ R_{0} $ is the radius of the observer and $ \theta $ is measured with respect to the line of sight to the observer.
\begin{figure}[h]
\begin{center}
\includegraphics[scale=0.5]{Figures/Flux.png}
\caption{ A sketch of the various angles and distances for the large angle (or high
latitude) emission when the $\gamma$ -ray source turns off suddenly.}
\end{center}
\end{figure}
The time dependence of observed flux follows from the lorentz transformation of
specific intensity. The specific flux in the observer frame from the relativistic source of moving object with specific intensity $ I_{\nu'} $ and spectrum frequency $\propto $ $ \nu'_{-\beta} $ is given by
\begin{equation}
 f_{\nu}(t_{obs})= \int d\Omega_{obs} I_{\nu}cos\theta_{obs}
 \end{equation}
 where  $ d\Omega_{obs}$ is the solid angle of the source, $ I_{\nu} $  is the specific intensity of the source photon. To derive the standard flux decay of GRBs, let we define $ d\Omega_{obs}$   and $ I_{\nu} $ in the relativistic beaming.\\\\ In relativistic beam of photons, the transverse component of the momentum does not change under Lorentz transformation,i.e its comoving and lab frame values are the same. Thus
\begin{equation}
\nu sin\theta = \nu'sin\theta'
\end{equation}
or
\begin{equation}
sin\theta =\frac{\nu'}{\nu}sin \theta'
\end{equation}
Since the photon frequency on the observer frame, $ \nu $, can be expressed in terms of the comoving frequency, $ \nu' $, using standard Lorentz transformation of photon as
\begin{equation}
\nu =\frac{\nu'}{\Gamma (1 -\frac{\upsilon cos\theta}{c})} =\nu'D
\end{equation}
where $ D $ is standard Doppler effect which is expressed as $ [\Gamma (1 -\frac{\upsilon cos\theta}{c} )]^{-1}$ . Then the ratio of the frequency become $ \frac{\nu'}{\nu} = \frac{1}{D}$ and substituting this ratio into Eq. (2.5), we obtain
\begin{equation}
sin\theta = \frac{sin\theta'}{D}
\end{equation}
For large $\Gamma$ , $\theta  \approx \frac{\theta'}{\Gamma}$. This tells us photons are focused in the forward direction such that the angular size of photo beam in the lab frame is smaller than it is in the comoving frame by a factor $\sim \Gamma $. And also the solid angle for a canonical beam of photons in lab frame is smaller than in the comoving frame by a factor of $\sim \Gamma^{2}$ . This implies the Lorentz transformation of solid angle is:
\begin{equation}
d\Omega =sin\theta d\theta d\phi =\frac{sin\theta'd\theta'd\phi'}{D^{2}}= \frac{d\Omega'}{D^{2}}
\end{equation}
The other parameter in the Lorentz transformation is the specific intensity. It is
defined as flux per unit frequency and solid angle carried by photos traveling with in a narrow conical beam with its axis perpendicular to surface dA. This means
\begin{equation}
I_{\nu} =\frac{dE}{d\nu dt_{obs}dAd\Omega}
\end{equation}
Considering $ d\nu'dt'{obs}dA' $ = $ d\nu dt_{obs} dA $, are Lorentz invariant and using Eq. (2.8) and $ E=\Gamma E' $ , Eq. (2.9) can be reduced to
\begin{equation}
I_{\nu} =D^{3}I'_{_{\nu'}}
\end{equation}
Since for intrinsic spectrum, $ I'_{\nu'}= I'\nu'^{-\beta} $ ,where $\beta $ is spectral index, then the specific intensity is summarized as
\begin{equation}
I_{\nu}= D^{3}\nu'^{-\beta}I'
\end{equation}
This equation, can be simplify by substituting the value of $\nu'$ from the Eq. (2.6)
\begin{equation}
I_{\nu} = D^{3+\beta}\nu^{-\beta}I'
\end{equation}
Finally substituting Eq. (2.8) and Eq. (2.12) into Eq. (2.3) and integrating
over  $ d\phi $ in the interval   $ 0  -  2 \pi $  , the observed flux becomes,
\begin{equation}
f_{\nu}(t_{obs}) = 2\pi \int d\theta_{obs}\frac{I'_{ \nu } \nu_{0}^{\beta}sin2\theta_{obs}[(1+z)\Gamma]^{-(3+\beta)}}{2\nu^{\beta}(1-\upsilon cos(\theta+\theta_{obs})/c)^{3+\beta}}
\end{equation}
where $ \nu'_{0} $ is the  frequency that lies on the power law segment of the spectrum for $ I'_{\nu'} $ . Using the law of sine from the diagram in Fig. 2.8, we see that $sin\theta/dA =sin\theta_{obs}/R_{0} $, this implies $ sin\theta_{obs}= \frac{R_{0}sin\theta}{dA}$ and substituting into Eq. (2.13) ,in the case $ \theta _{obs} \ll  \theta $, yields
\begin{equation}
f_{\nu}(t_{obs})=\frac{2 \pi I'_{0}\nu'_{0}\nu^{-\beta}}{[(1+z)]^{3+\beta}}(R_{0}/dA)^{2}\int_{\theta t}^{\pi/2} d \theta\frac{sin\theta cos\theta}{(1-\upsilon cos\theta/c)^{3+\beta}}
\end{equation}
Using substitution method of integrating, this equation can be simplified as
\begin{equation}
f_{\nu}(t_{obs})\propto [(1 -\frac{\upsilon cos\theta_{t}}{c} )]^{-(2+\beta)}\nu^{-\beta}
\end{equation}
Photons released at ($ r= vt $ , $ \theta $, $ \varphi $ ) arrive at the observer frame with a time delayed to  a photon emitted at $ r = 0 $ of
\begin{equation}
t_{obs}= t-\frac{\upsilon\cos\theta}{c}= t(1-\frac{\upsilon cos\theta}{c})=t/\Gamma D
\end{equation}
From this equation, the relation between $ t_{obs} $ and D is $t_{obs}  \propto D^{-1} $ . Then the flux decay with time of the observed light curves from Eq. (2.15) is summarized as:
\begin{equation}
f_{\nu}(t_{obs})\propto t^{-(2+\beta)}\nu^{-\beta}
\end{equation}
The standard convection of flux decay is
\begin{equation}
f_{\nu}(t_{obs}) \propto t^{-\alpha}\nu^{-\beta}
\end{equation}
where $ \alpha = 2 + \beta $
\section{calculating luminosity (L) of x-ray afterglow}
Luminosity is the total amount of electromagnetic energy radiated (out put) by an  object per unit of time. The observed isotropic-equivalent luminosity in the X-ray afterglow ,$ L_{x} $ can generally be expressed as 
\begin{equation}
L_{x}(t)= \int_{\nu_{1}}^{\nu_{2}} L_{\nu}(t)d\nu
\end{equation}
 where $ L_{\nu}(t)=\frac{4 \pi d_{L}^{2} F}{(1+z)} $,substituting  $L_{\nu}(t)$ in to equa. (2.4) reveals 
 \begin{equation}
 L_{x}(t)=\frac{4 \pi d_{L}^{2}}{(1+z)}\int_{\nu_{1}}^{\nu_{2}}\frac{F_{_{\nu}}}{(1+z)[(1+z)t]d\nu} 
 \end{equation}
where $ d_{L} $ is the luminosity distance, $\nu_{1}$  and $ \nu_{2} $ are the spectral frequencies in the energy band, z is the red shift and $ L_{\nu}  (t) $ is the spectral luminosity at the cosmological frame of the source, i.e, both $\nu $ and t are  measured in the frame [62].
\begin{equation}
L_{x}(t) = 4\pi d_{L}^{2}\int_{\nu_{1}/(1+z)}^{\nu_{2}/(1+z)}F_{\nu}[(1+z)t]d\nu
\end{equation}
since $F_{\nu}  (t) $ is measured in the observer frame and assumed in standard form, Eq. (2.21) can be reduced to:
\begin{equation}
L_{x}(t) = 4\pi d_{L}^{2}(1+z)^{(\beta-\alpha -1)} F_{x}(t)
\end{equation}
where $ F_{x}(t)  = \int _{\nu_{1}} ^{\nu_{2}}F_{\nu}(t)$ .
Here, we have understood that the flux decay of afterglow light curves are governed by standard power law of decay, i.e $ f _{\nu}(t_{obs})  \propto t^{-\alpha} \nu^{-\beta} $,where $ \alpha = 2+\beta $. This is a theoretical understanding for the afterglow era. Therefore , swift observation have led to  better understanding of afterglow  x-ray light curves for the initial few hours. The two mechanism of emissions have related to the behavior of central engine of the burst. Now let us introduce the method and material used to  analyze the temporal and spectral indices of canonical x-ray  afterglow  light curve in the next chapter.\\

