\setcounter{chapter}{2}
\chapter{Research methodology}
\label{chap:3}
\textbf{Introduction}\\
After the launch of the Swift , observational and theoritical understanding of the  prompt and afterglow phases of gamma-ray bursts promptedly changed due to use of satellites equipped with improved detecting instruments. Furthermore,the debating issues at early dicovery: the sources , the cosmological origin and isotropic distributions of GRBs were confirmed in swift era. Not only these , standared models " fire ball model " developed to explain the emmission mechanisms of gamma-ray burst and its afterglows ( from X-ray to radio band ).However,to achieve objectives of this study , un explained ideas and knowledge gaps that appeared in review litrature shouldbe answerd.Therefore,appropriate and  comperhensive methodology i.e , quantitative and qualitative research approaches and procedures are implemented.  
\section{Research designs}
\textbf{Modeling}\\
As we have mentioned in section (2.2), the  standard fireball model proposed to explain afterglow gamma-ray bursts. In standard fireball model, the behavior of X-ray light curves is assumed to be a single power law decay where flux fads  as: 
\begin{equation}
f_{\nu}(t)\propto  t^{-\alpha} 
\end{equation}
where  $f_{\nu} $the flux decay with time and  $ \alpha  $ is the temporal index/decay slope and subscripted by numbers  $ \alpha $ = 1, 2, 3,and 4 for early steep decay slope, shallow decay slope, normal decay slope and late decay slope respectively , that were captured by the swift/XRT. This  is the model that relates both temporal($ \alpha $) and spectral( $ \beta $ ) indices  in standard fireball model as:
 $\alpha$ = 2 + $\beta$  called the closure relation , where   both  $\beta $  and  $ \alpha $   are  unitless.\\\\
\textbf{Simple emperical model }\\ 
\section{Data sources,sampling techniques and  size }
In our synthesis of swift XRT light curves, I have been selected 10 GRBs as a represetative sample from which 5 are long GRBs and the other 5  are short GRBs with known red shifts and have  one or two light curve breaks for analyzing deta.The selection criterions were number of light curve breaks , types of gamma-ray bursts  and known redshifts.
\subsection{Data sources and types.}
I use secondary data from Evans et al online repository(swift /xrt ) already collected during observations over longer time.
\subsection{Data sampling technique and size}
For our data analysis 10 GRBs are  selected  using simple random probability sampling method. 
\section{Validity and reliability of data }
\section{Data processing and analyzing }
\section{Evaluating and justifying methodology}

 

