\setcounter{chapter}{2}
\chapter{Research methodology}
\label{chap:3}
\textbf{Introduction}\\
Here the  chapter, comprises  the main issue of the study. I describe the  research design / approach  that  appropriate  to  thesis title , target population , sampling techniques and sample size , models / tools and softwares used to analyze and interpret the data collected. I discussed briefly each procedure in subsequent sections as mentioned below.
\section{Research design/approach}
In this  section  of the thesis , I have described chosen research design,sample and techniques of sampling the data ,and  also models or tools that meet  the objective of the thesis  that mentioned in section 1.5 above.    that were used to analyze  early afterglow GRBs.(That is the characterstics of canonical X-ray light curve in Swift era ).Since its  discovery in 2005 , the afterglow light curves were analyzed by different researcher in different times in the swift era. Inorder to understand more about the temporal and spectral decay indices, the following directive methods, models and tools are used to analyze the data.

\section{Data Sources and types of data collected }
In our synthesis of swift XRT light curves, GRBs have been selected in which they
have more than two light curve breaks and specific or known redshifts. From the
Evans et al online repository, we have considered GRB data taken over the period of
six month in this work, which were triggered from January 2005 - July 2005. Among
40 GRBs that were triggered during this period, we have chosen a representative
sample of 4 GRBs for our analysis. The selection criterions were number of light
curve breaks and known redshift.\\\\Swift's X-Ray Telescope (XRT) is designed to measure the fluxes, spectra, and lightcurves of GRBs and afterglows over a wide dynamic range covering more than seven orders of magnitude in flux. The XRT can pinpoint GRBs to 5-arcsec accuracy within 10 seconds of target acquisition for a typical GRB and can study the X-ray counterparts of GRBs beginning 20-70 seconds from burst discovery and continuing for days to weeks. The layout of the XRT is shown in the schematic below; the table after it summarizes the instrument's parameters. Further information on the XRT is given by Burrows et al. (2000) and Hill et al. (2000).
\section{Data sampling technique and sample size}
\section{Data  processing , validity and reliability }
\subsection{Models/Tools used for analyzing and interpreting}
In this chapter we approach some classic problems of photometric analysis. We start with the light curves of eclipsing binary stars. These reduce, in their simplest form, to regular patterns of variation which can be understood by reference to relatively simple models of stars in a simple geometrical arrangement. First estimates for key parameters can often be directly made from inspection of the salient features of a light curve. This is a useful preliminary to more detailed analysis. The main issue underlying this and subsequent chapters, however, concerns the setting out of a comprehensive procedure for parameter value estimation. This represents a subsection of the field of optimization analysis, or the optimal curve-fitting problem.\\
In the version of this problem that faces us, we are given a set of N discrete observations lo (ti), i = 1, …, N, in a data space, which have a probabilistic relationship to an underlying physical variation, dependent in a single-valued way on time t. This real variation, whatever its form, is approximated by a fitting function, lc(aj, t), say, which is, formally, some function of the independent variable t, and a set of n parameters aj, j = 1, …, n. In general, we regard a subset m of these parameters as determinable from the data.

The object is to transfer the lo(ti) information from data space to the aj information in parameter space.
As mentioned in section (2.2), the  standard fireball model assumed to explain afterglow gamma-ray bursts. In standard fireball model, the behavior of X-ray light curves is assumed to be a single power law decay where flux goes as:
\begin{equation}
f_{\nu}(t)\propto  t^{-\alpha} 
\end{equation}
where  $f_{\nu} $the flux decay with time and  $ \alpha  $ is the temporal index/decay slope and subscripted by numbers  $ \alpha $ = 1, 2, 3,and 4 for early steep decay slope, shallow decay slope, normal decay slope and late decay slope respectively.As mentioned above,this model  indicated the closure relation of temporal and spectral indices  in standard fireball model as:
 $\alpha$ = 2 + $\beta$ where , $\beta $ stands for spectral index.\\\\  
\subsection{Simple emperical / Numerical approach}
\section{Evaluation and justification of  methodology }
