\section*{}
Although many X‐ray afterglows follow a “canonical” steep‐shallow‐standard decay pattern, some break the mould. The start of the Swift X‐ray light‐curve of GRB 080307 showed an unusual smooth rise, at the beginning of which the emission softened. After this brightening, the emission followed a simple power‐law decay, with no requirement for breaks. It is conjectured that the early softening is related to the tail of the prompt emission, which then fades rapidly away, allowing the rise of the afterglow to be seen. The optical afterglow was briefly detected by Gemini, Faulkes Telescope South and UKIRT, and the host galaxy by WHT, though no redshift was determined.
