\section*{}
We present new observations of the early X-ray afterglows of the first 27
gamma-ray bursts (GRBs) detected with the Swift X-ray Telescope (XRT). The
early X-ray afterglows show a canonical behavior, where the light curve broadly
consists of three distinct power law segments: (i) an initial very steep decay
( $\propto  t^{-\alpha} $  with 3 $\lesssim $  $\alpha_{1} $ $\lesssim $ 5) , followed by (ii) a very shallow decay ( 0.2 $\lesssim$  $ \alpha_{2} $ $\lesssim $ 0.8) , and finally (iii) a some what steeper decay (1 $\lesssim  \alpha_{3} \lesssim $1.5). These power law segments are separated by two corresponding break times, 300 s $\lesssim $  $ t_{break,1} $  $\lesssim $ 500 s and $ 10^{3} $ s  $\lesssim $ $ t_{break,2} $t $\lesssim $ $ 10^{4} $ s.\\\
 On top of this canonical behavior of the early X-ray light curve, many events have superimposed X-ray flares, which are most likely caused by internal shocks due to long lasting sporadic activity of the central engine, up to several hours after the GRB. We find that the initial steep decay is consistent with it being the tail of the prompt emission, from photons that are radiated at large angles relative to our line of sight.\\\ The first break in the light curve ( $ t_{break,1} $ ) takes place when the forward shock emission becomes dominant, with the intermediate shallow flux decay ($ \alpha_{2} $ ) likely caused by the continuous energy injection into the external shock. When this energy injection stops, a second break is then observed in the light curve ( $ t_{break,2} $  ). This energy injection increases the energy of the afterglow shock by at least a factor of f > 4 , and augments the already severe requirements for the efficiency of the prompt gamma-ray emission.
Subject headings: gamma rays: bursts — radiation mechanisms: non thermal